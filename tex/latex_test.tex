% source: https://medium.com/rahasak/vim-as-my-latex-editor-f0c5d60c66fa

\documentclass{article}

\title{Rahasak}
\author{SCoRe Lab @ university of colombo school of computing}
\date{July 2017}

\begin{document}


\maketitle   % the original file had this placed before `begin{document}`. It should be after

\section{Rahasak}

\subsection{About Rahasak}

Rahasak is an innovative application that helps your to share your secret data
such as voice calls, texts, location information and selfies without revealing
to the rest of the world. It never saves your secrets on its server. Rahasak 
encrypts your secret information between you and your friend, so no one can 
see them.

\subsection{Rahasak features}

\begin{enumerate}
\item Secret calls 
\item Secret texts
\item Secret location sharing
\item Secret selfies :)
\end{enumerate}

\subsection{Rahasak download}

Rahsak app can be downloaded from google play 
\newline
\href{https://play.google.com/store/apps/details?id=com.score.rahasak&hl=en}

\section{Security Architecture}
Rahsak app encrypts the communication by using AES 256 keys. AES key is 
exchanged between the users by using 1024 bit RSA public key enctyption. RSA 
public keys will be distributed to the users by using dual channels that is 
SenZ Switch and SMS. 

If someone wants to break the security of the system, both mobile SMS service 
provier and SenZ switch need to be simultaneously compromised. It is unlike to 
be happend and therefore Rahasak provides perfect data protection to their 
users.

\section{Technologies}

We mainly forcused on functional programming when building the services. 
Following is the technology stack

\begin{enumerate}
\item Scala 
\item Akka 
\item Reactive streaming
\item Erlang 
\item Android
\item Opus codec for audio compression
\end{enumerate}

\end{document}
